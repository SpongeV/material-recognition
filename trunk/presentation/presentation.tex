\documentclass{beamer}
\usepackage{beamerthemesplit}
\usetheme{CambridgeUS}%\usetheme{Frankfurt}
%\usetheme{Laly}
\usepackage{verbatim}
\usepackage{multirow}
\usepackage{amsmath} 
\usepackage{colortbl} 
\usepackage[noend]{algorithmic}
\usepackage{algorithm}
\usepackage{epsfig}
\usepackage{color}
\usepackage{hyperref}
%\definecolor{pacificblue}{RGB}{59,110,143}
\newcommand{\blue}[1]{\textcolor{pacificblue}{\textbf{#1}}}
\newcommand{\todo}[1]{\textcolor{red}{\textbf{#1}}}
\newcommand{\myemph}[1]{{\it #1}}
\newcommand{\mybf}[1]{{\bf #1}}
\fontfamily{georgia}\selectfont\normalsize
\newcommand{\ghline}[0]{\arrayrulecolor[rgb]{0.635,0.635,0.635} \hline}
\newcommand{\tspace}{\rule{0pt}{2.6ex}}
%==================================================================================================================================

\setbeamertemplate{footline}[frame number]
\title{Texture Synthesis for Material Recognition\\
\normalsize{Master\rq s Thesis in Articial Intelligence --- Intelligent Systems}}
\author{Jasper van Turnhout\\Student no. 0312649\\jturnhou@science.uva.nl}
\date{November 25, 2011}
\begin{document}
\frame{\titlepage}
\section[Outline]{}
\frame{\tableofcontents}

%==================================================================================================================================
\section{Introduction}
\frame{
	\frametitle{Material Recognition}
	\begin{itemize}
		\item The task of classifying single novel images to material classes 
		\item Material models largely dependent on the intra-class variation of training data
		\item Data can be sparse/hard to obtain
	\end{itemize}
	ADD IMAGE
}
\frame{
	\frametitle{Texture Synthesis}
	\begin{itemize}
		\item The task of creating synthetic images
		\item Different reflection models simulate different light behavior
		\item Diffuse reflection models to generate diffuse surfaces
		\item Specular reflection models to generate shiny/glossy surfaces
	\end{itemize}
	ADD IMAGE
}
\frame{
	\frametitle{Goal of this thesis}
	\begin{itemize}
		\item Investigate how synthetic image data can support the field of material recognition
		\item What reflection models can be employed to obtain highly realistic synthetic data
		\item What are potential bottlenecks in the creation and usage of synthetic data
	\end{itemize}
}
%==================================================================================================================================
\section{Related Work}
\frame{
	\frametitle{Textons \& Filter Banks}
	\begin{itemize}
		\item Different filter banks are used to obtain image responses
		\item A pixel in an image becomes a vector describing bumps, ridges, edges and illumination properties such as specularity
		\item A texton dictionary is formed by clustering the vectors: each cluster center represents a texton
		\item Material models are obtained by labeling the pixels to the nearest textons
		\item Novel images are quantized to texton representation. 
	\end{itemize}
}
\frame{
	\frametitle{Multivariate Gaussian Distributions}
	\begin{itemize}
		\item Investigage how synthetic image data can support the field of material recognition
	\end{itemize}
}
\frame{
	\frametitle{Minimal Training Images}
}

%==================================================================================================================================
\section{Approach}
\frame{
	\frametitle{Photometric Stereo}
}
\frame{
	\frametitle{PhoTex Database}
}
\frame{
	\frametitle{Generation of novel data}
}

%==================================================================================================================================
\section{Fundamentals}
\frame{
	\frametitle{Local Reflection}
}
\frame{
	\frametitle{Lambert's Cosine Law}
}

%==================================================================================================================================
\section{Reflection Models}
\frame{
	\frametitle{Lambertian}
}
\frame{
	\frametitle{Phong}
}
\frame{
	\frametitle{Blinn-Phong}
}
\frame{
	\frametitle{Cook-Torrance}
}
\frame{
	\frametitle{Oren-Nayar}
}
%==================================================================================================================================
\section{Experiments}
\frame{
	\frametitle{Two datasets}
}
\frame{
	\frametitle{Experiment A}
}
\frame{
	\frametitle{Experiment B}
}
\frame{
	\frametitle{Results}
}

%==================================================================================================================================
\section{Conclusion}
\frame{
	\frametitle{Conclusion}
}


%==================================================================================================================================
\end{document}
%==================================================================================================================================




